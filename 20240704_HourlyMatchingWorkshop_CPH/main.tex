% !TeX root = main.tex

\input{backmatter.tex}

\title{Insights from model based studies \\ on 24/7 CFE and green hydrogen regulation}

%\subtitle{---}
\author{
  Elisabeth Zeyen \& Iegor Riepin\\
  \hrefc{mailto:e.zeyen@tu-berlin.de}{e.zeyen@tu-berlin.de} $\vert\vert$ 
  \hrefc{mailto:iegor.riepin@tu-berlin.de}{iegor.riepin@tu-berlin.de} \\
  Technical University of Berlin
  }

\date{DTU, 04 July 2024 \\

}

\titlegraphic{%
  \vspace{0cm}
  \hspace{10.7cm}
    \includegraphics[trim=0 0cm 0 0cm,height=1.2cm,clip=true]{images/TUB.png}
  \vspace{5.5cm}
  }
  
\begin{document}

\maketitle

%%%% 




% % Data centers work on technical solutions
% \begin{frame}{ICT companies work on concepts and technical solutions}

%   \begin{columns}[T]
%     \begin{column}{8cm}
%     \includegraphics[width=9.1cm]{images/Radovanovic_blog.png}
%   \end{column}

%   \begin{column}{7cm}
%     \includegraphics[width=7cm]{images/Koningstein_blog.png}
%     \source{
%       Sources: \\
%       \hrefc{https://blog.google/inside-google/infrastructure/data-centers-work-harder-sun-shines-wind-blows/}{blog.google/data-centers-work-harder-sun-shines-wind-blows} \\
%       \hrefc{https://blog.google/outreach-initiatives/sustainability/carbon-aware-computing-location/}{blog.google/carbon-aware-computing-location}}
% \end{column}

% \end{columns}
% \end{frame}



% New study released: here we explore the value of demand flexibility for 24/7 CFE
\begin{frame}{New study: The value of space-time load-shifting flexibility \\ 
  for 24/7 carbon-free electricity procurement (July 2023)}
  
    {\small
    \begin{columns}
    \begin{column}{9cm}
    \includegraphics[width=10cm]{images/spatial-temporal-vlinks-cropped.png}
    \end{column}
  
    \begin{column}{6cm}
  
    \begin{itemize}
      \vspace{-0.2cm}
       \item Key focuses: \\
     -- How can demand flexibility reduce the required \alert{resources} and \alert{costs} of 24/7~CFE matching?\\ 
     \vspace{0.1cm}
     -- What are the \alert{signals} for optimal utilisation of demand flexibility?\\
     \vspace{0.1cm}
     -- What are the trade-offs and synergies from co-optimisation of \alert{spatial} and \alert{temporal} load shifting?
  
     \item Open-access research: \\
     \vspace{0.2cm}
     {\footnotesize
     \faUnlock~study:
     \hrefc{https://zenodo.org/records/8185850}{zenodo.org/records/8185850}\\
     \faUnlock~code:
     \hrefc{https://github.com/PyPSA/247-cfe}{github.com/PyPSA/247-cfe}\\
     }

     \item  A follow-up research paper to be released in March 2024.
  
    \end{itemize}
    \end{column}
    \end{columns}
    }
  \end{frame}
  
  
  % Brief overview of the study design
  \begin{frame}{Methods and study design}
  
    {\footnotesize
    \begin{columns}
    \begin{column}{9cm}
    \includegraphics[width=10cm]{images/spatial-temporal-vlinks-cropped.png}
    \end{column}
  
    \begin{column}{6cm}
      \begin{itemize}
        \vspace{-0.1cm}
        \item The study is done with \alert{PyPSA} -- an open-source framework for modelling modern energy systems.
        \item Model scope: \alert{\href{https://www.entsoe.eu/data/map/}{ENTSO-E area}} power system clustered to individual bidding zones, \alert{hourly} temporal resolution. 
        \item Geographically scattered datacenters that are managed collectively. An operating company follows \alert{24/7 CFE strategy} in all locations.
        \item \alert{Spatial} and \alert{temporal} load shifting mechamisms.
        \item \alert{\enquote{Flexible workloads}}, i.e. electricity loads that can potentially be shifted in space or in time, are assumed to be in a range of {\{0\% .. 40\%\}}.
      \end{itemize}
      \end{column}
      \end{columns}
    }
\end{frame}
  

\begin{frame}{Signal 1: quality of local renewable resouces}

  \centering
  \vspace{0.3cm}
  \includegraphics[width=14cm]{images/results-1.png}

\end{frame}


\begin{frame}{Signal 2: low correlation of wind power generation over long distances}
  \centering
  \vspace{0.3cm}
  \includegraphics[width=14cm]{images/results-4.png}
\end{frame}


\begin{frame}{Cost savings as a function of distance between datacenter pair}
  \centering
  \vspace{0.3cm}
  \includegraphics[width=14cm]{images/results-5.png}
\end{frame}


\begin{frame}{Time-series of optimized spatial load shifts (locations: DK-IE)}
  \centering
  \vspace{0.3cm}
  \includegraphics[width=14cm]{images/results-6.png}
\end{frame}


\begin{frame}{Signal 3: time lag in solar radiation peaks due to Earth's rotation (1/2)}
  \centering
  \vspace{0.3cm}
  \includegraphics[width=14cm]{images/results-7.png}
\end{frame}


\begin{frame}{Signal 3: time lag in solar radiation peaks due to Earth's rotation (2/2)}
  \centering
  \vspace{0.3cm}
  \includegraphics[width=14cm]{images/results-8.png}
\end{frame}



\begin{frame}{Also in the study}
  
      \begin{itemize}
      \item Scenarios for \alert{co-optimised} and \alert{isolated} utilisation of space-time load-shifting;
      \item Scenarios for 24/7 CFE with \alert{98\% and 100\%} matching targets;
      \item Scenarios with different \alert{24/7 technology options} (e.g., Long Duration Energy Storage);
      \item 24/7 CFE \alert{cost breakdowns} and \alert{procurement strategies} for individual locations;
      \item \alert{Synergies} and \alert{trade-offs} between spatial and temporal load shifting;
      \item Analysis of \alert{net load migration} across locations;
      \item Simulated \alert{energy balances} for selected datacenters.
      \end{itemize}

\end{frame}

\begin{frame}{Take aways}

There are \alert{three signals} companies can factor into their procurement \& load shaping strategies for 24/7 CFE matching: \\
\vspace{0.1cm}
  -- quality of local renewable resources;\\
  -- low correlation of wind power generation over long distances;\\
  -- time lag in solar radiation peaks due to Earth's rotation.\\

\vspace{0.5cm}

Overall, space-time load-shifting flexibility: \\
\vspace{0.1cm}
  -- enables \alert{better access to clean electricity} and creates \alert{more options} for consumers to match demand with carbon-free electricity around-the-clock; \\
  -- \alert{lowers the costs} of 24/7 CFE matching and makes it \alert{more attractive} to a wider range of companies.\\

    
\end{frame}


%----------------------------------------
%----------------------------------------

\begin{frame}\frametitle{\quad}

  {\Large
  \alert{Contacts, Resources, Acknowledgements}
  }

  \vspace{0.1cm}
    
  {\bf References:}
  \hrefc{https://iopscience.iop.org/article/10.1088/1748-9326/ad2239}{Temporal regulation of renewable supply for electrolytic hydrogen (2023)}\\
  {\bf References:} \hrefc{https://irieo.github.io/247cfe.github.io/}{More about the 24/7 CFE research project (2022-2024)}

  {\bf Code:} This work done in a spirit of open and reproducible research: \\
  \faUnlock~code:
  \hrefc{https://github.com/PyPSA/247-cfe}{github.com/PyPSA/247-cfe} \\
  \faUnlock~code: \hrefc{https://zenodo.org/records/8324521}{https://zenodo.org/records/8324521}
  
  \vspace{.1cm}
  {\bf Copyright:} Unless otherwise stated, graphics and text are Copyright \copyright E.Z. and I.R. 2024. \\
  This work is licensed under a \href{https://creativecommons.org/licenses/by/4.0/}{CC BY 4.0}.  {\footnotesize \ccby} 

  \vspace{.1cm}
  {\bf Send an email:} \\
  Dr. Elisabeth Zeyen, e.zeyen@tu-berlin.de\\
  Dr. Iegor Riepin, iegor.riepin@tu-berlin.de \\
  
\end{frame}


%%%%%%%%%%%%%%%%%%%%%%%%%%%%%%%%%%%%%%%%%%%%%%%%%%%%%%
\end{document}
%%%%%%%%%%%%%%%%%%%%%%%%%%%%%%%%%%%%%%%%%%%%%%%%%%%%%%