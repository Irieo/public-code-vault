% !TeX root = main.tex

\input{backmatter.tex}

\usepackage[
type={CC},
modifier={by},
version={4.0},
]{doclicense}

\title{Three Pillars of Hourly Matching: \\ A Tour From Model Land to Real World}

%\subtitle{---}
\author{
  Elisabeth Zeyen \& Iegor Riepin\\
  \hrefc{mailto:e.zeyen@tu-berlin.de}{e.zeyen@tu-berlin.de} $\vert\vert$ 
  \hrefc{mailto:iegor.riepin@tu-berlin.de}{iegor.riepin@tu-berlin.de} \\
  Technical University of Berlin
  }

\date{EURO2024, Hydrogen and Electricity Modeling and Regulation\\ 
      03 July 2024 \\

}

\titlegraphic{%
  \vspace{0cm}
  \hspace{10.7cm}
    \includegraphics[trim=0 0cm 0 0cm,height=1.2cm,clip=true]{images/TUB.png}
  \vspace{5.5cm}
  }


\setbeamertemplate{footline}{
	\usebeamercolor[fg]{framesource}%
	\usebeamerfont{page number in head}%
	\hspace{0.2cm}
	\small \insertframenumber
	\vspace{0.2cm}
	\scalebox{0.5}{\doclicenseIcon}
	\hfill
%	\includegraphics[height=0.8cm]{images/tublogo.pdf}
%	\hspace{0.2cm}
}

% Disable the section title slides
\AtBeginSection[]{}
\setbeamercovered{transparent}

\setbeamertemplate{footline}[
myframe number]

% Change alert color to black
\setbeamercolor{alerted text}{fg=black}

% SOURCES
\setbeamercolor{framesource}{fg=gray}
\setbeamerfont{framesource}{size=\tiny}


% Redefine \sectionpage to only show the section name and not increment the frame counter
\makeatletter
\patchcmd{\sectionpage}{\usebeamertemplate*{headline}}{}{}{}
\patchcmd{\sectionpage}{\begin{center}\usebeamerfont*{section name}\insertsectionnumber.\ \end{center}}{}{}{}
\patchcmd{\sectionpage}{\begin{center}\usebeamerfont*{section title}\usebeamercolor[fg]{section title}\insertsectionhead\end{center}}{\begin{center}\usebeamerfont*{section title}\usebeamercolor[fg]{section title}\insertsectionhead\end{center}}{}{}
\makeatother

% Redefine \sectionpage to remove the section number
\def\sectionpage{\begin{centering}
		\begin{beamercolorbox}[sep=12pt,center]{part title}
			\usebeamerfont{part title}\insertsectionhead\par
		\end{beamercolorbox}
	\end{centering}
}

\usetheme[]{Berlin}
\begin{document}

\maketitle

%%%% 

\section{Introduction}

\begin{frame}{The Three Pillars}

  \begin{columns}
  	\begin{column}{0.6\textwidth}
  		 \begin{itemize}
  			\item A concept of \alert{hourly matching} got into the spotlight with debates on clean hydrogen regulation
        \item Also a foundation for voluntary 24/7 carbon-free electricity (CFE) procurement  
  			\item Set of rules and regulations, aka \alert{the Three Pillars} address aspects of temporal alignment, geographical scope, and additionality of clean energy supply
  			\item Today, let's: \\
        -- discuss the Three Pillars in the real world and in the mathematical models\\
        -- try to bridge the gap between the two
        \end{itemize}        
  	\end{column}
  \begin{column}{0.4\textwidth}
  		\includegraphics[width=1\linewidth]{images/3pillars}  
  		\source{ChatGPT4 hallucination, 2024}	
  \end{column}
  \end{columns}

\end{frame}


\section{I Pillar - Temporal matching}
\begin{frame}{I Pillar: Temporal matching}
	\textbf{Discussion}: Implementing stricter regulations requires assessing the impact on hydrogen production costs.
	\begin{columns}[t]
		\begin{column}{0.5\textwidth}
			\centering
				\includegraphics[width=5cm]{images/h2_temporal_news4}
				\newline
				\vspace{0.2cm}
				\newline
				\includegraphics[width=4cm]{images/h2_temporal_news1}
		\end{column}
	\begin{column}{0.5\textwidth}
		\centering
		\includegraphics[width=5cm]{images/h2_temporal_news2}
		\newline
		\vspace{0.4cm}
		\centering
		\newline
		\includegraphics[width=5cm]{images/h2_temporal_news3}
		
	\end{column}
	\end{columns}
\source{\href{https://www.hydrogeninsight.com/policy/far-from-perfect-strict-rules-in-new-delegated-act-will-make-green-h2-projects-more-expensive-hydrogen-europe/2-1-1403241}{Hydrogeninsight (February 2023)}, \href{https://www.abnamro.com/research/en/our-research/sustainaweekly-not-so-green-hydrogen-perhaps-a-necessary-evil}{Abnamro (October 2022)}, \href{https://www.rechargenews.com/energy-transition/proposed-stringent-eu-rules-on-green-hydrogen-would-put-the-brakes-on-development-/2-1-1223746}{rechargenews (May 2022)}, \href{https://www.hydrogeninsight.com/policy/far-from-perfect-strict-rules-in-new-delegated-act-will-make-green-h2-projects-more-expensive-hydrogen-europe/2-1-1403241}{Hydrogeninsight (February 2023)}}
\end{frame}
\begin{frame}{Temporal matching: definition}
	
	\alert{Hourly matching} is modelled with a constraint, 
	which matches electricity demand for electrolysis process with generation of renewable resources on an hourly basis. 
	\centering
	\includegraphics[width=0.6\linewidth, clip, trim={0cm 0cm 0cm 0cm}]{images/hourly_eq_explained_v0}

	\source{\href{https://iopscience.iop.org/article/10.1088/1748-9326/ad2239}{Zeyen et al. (2023): Temporal regulation \\ of renewable supply for electrolytic hydrogen}}

\end{frame}
\begin{frame}{Temporal matching: definition}
	\addtocounter{framenumber}{-1}
	\alert{Hourly matching} is modelled with a constraint, 
	which matches electricity demand for electrolysis process with generation of renewable resources on an hourly basis. 
	\centering
	\includegraphics[width=0.6\linewidth, clip, trim={0cm 0cm 0cm 0cm}]{images/hourly_eq_explained_v1}
	
	\source{\href{https://iopscience.iop.org/article/10.1088/1748-9326/ad2239}{Zeyen et al. (2023): Temporal regulation \\ of renewable supply for electrolytic hydrogen}}

\end{frame}
\begin{frame}{Temporal matching: definition}
	\addtocounter{framenumber}{-1}
	\alert{Hourly matching} is modelled with a constraint, 
	which matches electricity demand for electrolysis process with generation of renewable resources on an hourly basis. 
	\centering
	\includegraphics[width=0.6\linewidth, clip, trim={0cm 0cm 0cm 0cm}]{images/hourly_eq_explained_v2}
	
	\source{\href{https://iopscience.iop.org/article/10.1088/1748-9326/ad2239}{Zeyen et al. (2023): Temporal regulation \\ of renewable supply for electrolytic hydrogen}}

\end{frame}
\begin{frame}{Temporal matching: definition}
	\addtocounter{framenumber}{-1}
	\alert{Hourly matching} is modelled with a constraint, 
	which matches electricity demand for electrolysis process with generation of renewable resources on an hourly basis. 
	\centering
	\includegraphics[width=0.6\linewidth, clip, trim={0cm 0cm 0cm 0cm}]{images/hourly_eq_explained_v3}
	
	\source{\href{https://iopscience.iop.org/article/10.1088/1748-9326/ad2239}{Zeyen et al. (2023): Temporal regulation \\ of renewable supply for electrolytic hydrogen}}

\end{frame}
\begin{frame}{Temporal matching: definition}
\addtocounter{framenumber}{-1}
  \alert{Hourly matching} is modelled with a constraint, 
  which matches electricity demand for electrolysis process with generation of renewable resources on an hourly basis. 

%  \vspace{0.1cm}
%  \begin{equation}
%  \sum_{r\in R} g_{r,t} + \sum_{s\in S} \left(\bar{g}_{s,t} - \ubar{g}_{s,t}\right) - ex_t = d_t
%  \label{eqn:CFE_1}
%  \end{equation}
%  \vspace{0.1cm}
  \centering
  \includegraphics[width=0.6\linewidth, clip, trim={0cm 0cm 0cm 0cm}]{images/hourly_eq_explained}
%  \noindent\fbox{%
%    \parbox{\textwidth}{%
%  Here eq. 2 from https://iopscience.iop.org/article/10.1088/1748-9326/ad2239/pdf \\
%  -- maybe here we say along the lines that the main fight in the Delegated Act was for this equation. \\
%  -- explain the equation\\ 
%  -- Now, what are the hidden issues? let's talk about the first point -- flexibility. \\
%  }}
  \source{\href{https://iopscience.iop.org/article/10.1088/1748-9326/ad2239}{Zeyen et al. (2023): Temporal regulation \\ of renewable supply for electrolytic hydrogen}}
\end{frame}


\begin{frame}{Temporal matching: flexibility changes everything}
	\vspace{-5cm} 
	\centering
	\includegraphics[width=0.6\linewidth, clip, trim={0cm 5cm 0cm 5cm}]{images/store_flexibility_v2}


	\source{\href{https://iopscience.iop.org/article/10.1088/1748-9326/ad2239}{Zeyen et al. (2023): Temporal regulation \\ of renewable supply for electrolytic hydrogen}}
\end{frame}

\begin{frame}{Temporal matching: flexibility changes everything}
	\addtocounter{framenumber}{-1}
	\vspace{-1cm} 
	\centering
	\includegraphics[width=0.6\linewidth, clip, trim={0cm 5cm 0cm 5cm}]{images/store_flexibility_v2}
	\begin{columns}[t]
		\begin{column}{0.5\textwidth}
			\centering
				\textbf{Emissions} \\
				\includegraphics[width=1\linewidth, clip, trim={5cm 4cm 5cm 4cm}]{images/emissions_flexibility}
		\end{column}
		\begin{column}{0.5\textwidth}
				\textbf{Hydrogen production costs} \\
				\includegraphics[width=1\linewidth, clip, trim={5cm 4cm 5cm 4cm}]{images/costs_flexibility}
		\end{column}
	\end{columns}

	Flexibility \alert{reduces} emissions and costs.
	\source{\href{https://iopscience.iop.org/article/10.1088/1748-9326/ad2239}{Zeyen et al. (2023): Temporal regulation \\ of renewable supply for electrolytic hydrogen}}
\end{frame}

\begin{frame}{Temporal matching: higher excess sales reduce emissions and costs}
	\centering
	\vspace{-2cm}
	The \alert{costs of hydrogen production decrease} by up to 30\% in case of flexible demand + hourly matching due to the larger profits from excess sales.
	\vspace{0.5cm}
  	\begin{columns}[t]
  	\begin{column}{0.5\textwidth}
  		\centering
  		\textbf{Annual} \\
  		\includegraphics[width=1\linewidth]{images/annual_excess}
  	\end{column}
  	\begin{column}{0.5\textwidth}
  		\centering
  		\textbf{Hourly} \\
  		\includegraphics[width=1\linewidth]{images/hourly_excess}
  	\end{column}
  \end{columns}
  \source{\href{https://iopscience.iop.org/article/10.1088/1748-9326/ad2239}{Zeyen et al. (2023): Temporal regulation \\ of renewable supply for electrolytic hydrogen}}

\end{frame}


\begin{frame}{Temporal matching: considering grid CFE supply}

  Here, the \alert{hourly matching constraint} is extended to include electricity consumption from the regional grid:

  \vspace{0.1cm}
  \begin{equation*}
  \sum_{r\in CFE, t\in T} g_{r,t} + \sum_{s\in STO, t\in T} \left(\bar{g}_{s,t} - \ubar{g}_{s,t}\right) - \sum_{t\in T} ex_t + \textcolor{red}{\sum_{t\in T} CFE_t \cdot im_t} \geq x \cdot \sum_{t\in T} d_t
  \label{eqn:CFE}
  \end{equation*}
  \vspace{0.1cm}
  \source{Riepin \& Brown (2023): https://zenodo.org/records/7180098}

  \noindent\fbox{%
    \parbox{\textwidth}{%
    A \enquote{CFE score of the grid} $CFE_t$ can be seen as the percentage of clean electricity in each MWh of imported     electricity from the regional grid to supply participating 24/7 loads in a given hour.
    }}
    
    Challenges:\\
    -- Choosing a \alert{geographical scope} to calculate  $CFE_t$\\
    -- Considering a \alert{residual mix} \\
    -- $\sum_{t\in T} CFE_t \cdot im_t$ is a \alert{nonconvex term}

\end{frame}


\section{II Pillar - Geographical matching}
\begin{frame}{II Pillar - Geographical matching}
  \includegraphics[width=13cm]{images/go_leak.png}
  \source{https://www.ffe.de/en/publications/norway-and-the-double-marketing-of-renewable-energies/}
\end{frame}



\begin{frame}{Geographical matching}

  \includegraphics[width=8cm]{images/microsoft.png}
  \source{datacenterdynamics "Microsoft signs 24/7 nuclear power deal with Constellation for Boydton data center" (June 30, 2023)}

\end{frame}

\begin{frame}{Geographical matching}

  \includegraphics[width=13cm]{images/microsoft2.png}

\end{frame}
  

\begin{frame}{Geographical matching for green hydrogen production}
\begin{columns}
	\begin{column}{0.4\textwidth}
		\includegraphics[width=5cm, clip, trim={0 0.1cm 0 0}]{images/biding_zones_eu.png}
		\source{\href{https://acer.europa.eu/sites/default/files/documents/en/Documents/Presentations\%20Webinars/20210624_Public_webinar_alternative_BZ_configurations.pdf}{ACER, 2021}}
	\end{column}
	\begin{column}{0.6\textwidth}
		How far can renewable generation and electrolysis be apart? 
		\begin{itemize}
			\item 	Regulation in the EU: Geographical matching within one \alert{bidding zone}.
			\item 	Model implementation is straightforward. 
		\end{itemize}

	\vspace{0.5 cm}
		\textbf{Challenges}: Especially in large bidding zones there can be \alert{transmission bottlenecks} actually hindering the transport of renewable generation to the electrolysis. 
	\end{column}
\end{columns}
\end{frame}

\section{III Pillar - Additionality}
\begin{frame}{III Pillar - Additionality}

  \begin{itemize}
  
  \item  \textbf{Idea}: carbon-free energy supply enabling the deployment of clean electricity generation (or clean hydrogen production) should be new---\alert{and additional}---to the grid, i.e., it would \textit{not} have occurred without the action of the claiming party.

  \item \textbf{Real world}: Complicated, there is no conterfactual

  \item \textbf{Models}: Complex (really!)
  
  \end{itemize}

\end{frame}


\begin{frame}{Additionality: definition challenges}

\begin{itemize}
	\item \textbf{Voluntary clean electricity procurement}: \enquote{Generally speaking, Google seeks to sign contracts to purchase both electricity and carbon-free attributes from clean energy projects that are in the development stage} (Google, 2021).
	\item \textbf{EU Delegated Act for Renewable Hydrogen}: \enquote{Hydrogen producer need to conclude power purchase agreements (PPA)s with new and unsupported renewable generation capacities.} 
\end{itemize}
\textbf{Challenge}: Hard to define if renewables are truly additional or would have been built in any case to reach national renewable generation targets.\\
\textbf{NB}: What should be additional: capacity (MW) or energy (MWh)?
\end{frame}


\begin{frame}{Additionality: how to model a background grid 1/2?}

  \begin{columns}
    \begin{column}{7cm}
      \includegraphics[width=7.5cm]{images/10-2025-DE-p3-ci_emisrate.pdf}
    \end{column}
    \begin{column}{7cm}
      \includegraphics[width=7.5cm]{images/ci_emisrate.pdf}
    \end{column}
  \end{columns}
An illustrative example of 24/7 CFE matching in Germany 2025. Left: constrained RES expansion (aligned with the national plan), Right: unconstrained RES expansion 
\source{https://zenodo.org/records/7180098\\
https://zenodo.org/records/10407831}
\end{frame}


\begin{frame}{Additionality: how we model a background grid 2/2?}
\begin{columns}[t]
	\begin{column}{0.5\textwidth}
		\centering
		\textbf{Two step optimisation} \\
			Clear additionality
			\includegraphics[width=1\linewidth]{images/emissions_twostep_v2}		
	\end{column}
	\begin{column}{0.5\textwidth}
		\centering
		\textbf{One step optimisation} \\
		RES compete for good resources
			\includegraphics[width=1\linewidth]{images/onestepoptimisation}
	\end{column}
\end{columns}
 There are many studies who analysed the impact on how additionality is modelled in more detail, e.g. \href{https://www.nature.com/articles/s41560-023-01435-0}{Giovanello et al. (2024), \href{https://papers.ssrn.com/sol3/papers.cfm?abstract_id=4636218}{Langer et al. (2024)}}.
\end{frame}

\section{Take Aways}
\begin{frame}{Take aways}

\begin{columns}
\begin{column}{7cm}
  \begin{enumerate}
    \item Do not take results for granted, try to challenge everything! \\
    \enquote{We must fight our biases to see the solutions} (Auke Hoekstra)
    \item Talk to others in research community, share your concerns! \\
    \enquote{A problem shared is a problem halved} (Old saying)
  \end{enumerate}
\end{column}
\begin{column}{8cm}
  \includegraphics[width=8cm]{images/alice.png}
  \source{ChatGPT4 hallucination, 2024}
\end{column}
\end{columns}


\end{frame}

%----------------------------------------
%----------------------------------------

\begin{frame}\frametitle{\quad}

  {\Large
  \alert{Contacts, Resources, Acknowledgements}
  }

  \vspace{0.1cm}
    
  {\bf References:}
  \hrefc{https://iopscience.iop.org/article/10.1088/1748-9326/ad2239}{Temporal regulation of renewable supply for electrolytic hydrogen (2023)}\\
  {\bf References:} \hrefc{https://irieo.github.io/247cfe.github.io/}{More about the 24/7 CFE research project (2022-2024)}

  {\bf Code:} This work done in a spirit of open and reproducible research: \\
  \faUnlock~code:
  \hrefc{https://github.com/PyPSA/247-cfe}{github.com/PyPSA/247-cfe} \\
  \faUnlock~code: \hrefc{https://zenodo.org/records/8324521}{https://zenodo.org/records/8324521}
  
  \vspace{.1cm}
  {\bf Copyright:} Unless otherwise stated, graphics and text are Copyright \copyright E.Z. and I.R. 2024. \\
  This work is licensed under a \href{https://creativecommons.org/licenses/by/4.0/}{CC BY 4.0}.  {\footnotesize \ccby} 

  \vspace{.1cm}
  {\bf Send an email:} \\
  Dr. Elisabeth Zeyen, e.zeyen@tu-berlin.de\\
  Dr. Iegor Riepin, iegor.riepin@tu-berlin.de \\
  
\end{frame}


%%%%%%%%%%%%%%%%%%%%%%%%%%%%%%%%%%%%%%%%%%%%%%%%%%%%%%
\end{document}
%%%%%%%%%%%%%%%%%%%%%%%%%%%%%%%%%%%%%%%%%%%%%%%%%%%%%%