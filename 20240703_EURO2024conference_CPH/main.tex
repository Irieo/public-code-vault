% !TeX root = main.tex

\input{backmatter.tex}

\usepackage[
type={CC},
modifier={by},
version={4.0},
]{doclicense}

\title{Three Pillars of Hourly Matching: \\ A Tour From Model Land to Real World}

%\subtitle{---}
\author{
  Elisabeth Zeyen \& Iegor Riepin\\
  \hrefc{mailto:e.zeyen@tu-berlin.de}{e.zeyen@tu-berlin.de} $\vert\vert$ 
  \hrefc{mailto:iegor.riepin@tu-berlin.de}{iegor.riepin@tu-berlin.de} \\
  Technical University of Berlin
  }

\date{EURO2024, Hydrogen and Electricity Modeling and Regulation\\ 
      03 July 2024 \\

}

\titlegraphic{%
  \vspace{0cm}
  \hspace{10.7cm}
    \includegraphics[trim=0 0cm 0 0cm,height=1.2cm,clip=true]{images/TUB.png}
  \vspace{5.5cm}
  }


\setbeamertemplate{footline}{
	\usebeamercolor[fg]{framesource}%
	\usebeamerfont{page number in head}%
	\hspace{0.2cm}
	\small \insertframenumber
	\vspace{0.2cm}
	\scalebox{0.5}{\doclicenseIcon}
	\hfill
%	\includegraphics[height=0.8cm]{images/tublogo.pdf}
%	\hspace{0.2cm}
}

% Disable the section title slides
\AtBeginSection[]{}
\setbeamercovered{transparent}

\setbeamertemplate{footline}[
myframe number]

% Change alert color to black
\setbeamercolor{alerted text}{fg=black}

% SOURCES
\setbeamercolor{framesource}{fg=gray}
\setbeamerfont{framesource}{size=\tiny}


% Redefine \sectionpage to only show the section name and not increment the frame counter
\makeatletter
\patchcmd{\sectionpage}{\usebeamertemplate*{headline}}{}{}{}
\patchcmd{\sectionpage}{\begin{center}\usebeamerfont*{section name}\insertsectionnumber.\ \end{center}}{}{}{}
\patchcmd{\sectionpage}{\begin{center}\usebeamerfont*{section title}\usebeamercolor[fg]{section title}\insertsectionhead\end{center}}{\begin{center}\usebeamerfont*{section title}\usebeamercolor[fg]{section title}\insertsectionhead\end{center}}{}{}
\makeatother

% Redefine \sectionpage to remove the section number
\def\sectionpage{\begin{centering}
		\begin{beamercolorbox}[sep=12pt,center]{part title}
			\usebeamerfont{part title}\insertsectionhead\par
		\end{beamercolorbox}
	\end{centering}
}

\usetheme[]{Berlin}
\begin{document}

\maketitle

%%%% 

\section{Introduction}

\begin{frame}{Setting the scene}
  
  \begin{itemize}
    \item Voluntary 24/7 CFE procurement in electricity, Delegated act for H2 Regulation
    \item Both focus on the \enquote{Three Pillars} -- set of rules on temporal, spatial and additionality aspects of matching demand and clean power supply
    \item Today we go through the \alert{three pillars} and discuss: \\
    -- how the pillars are implemented in the real world \\
    -- how the pillars are implemented in mathematical models \\
    -- caveats and challenges of model implementations \\
    -- impact of model setups on results and policy recommendations \\
  \end{itemize}

  Iegor?
  
\end{frame}

\section{I Pillar - Temporal matching}
\begin{frame}{I Pillar: Temporal matching}
	\begin{columns}[t]
		\begin{column}{0.5\textwidth}
				\includegraphics[width=0.7\linewidth]{images/h2_temporal_news4}
				\href{https://www.hydrogeninsight.com/electrolysers/hourly-electricity-matching-is-the-only-reliable-way-to-reduce-emissions-from-green-hydrogen-study/2-1-1378431}{Hydrogeninsight, December 2022}
				\includegraphics[width=0.7\linewidth]{images/h2_temporal_news1}
				\href{https://www.abnamro.com/research/en/our-research/sustainaweekly-not-so-green-hydrogen-perhaps-a-necessary-evil}{Abnamro, October 2022}
		\end{column}
	\begin{column}{0.5\textwidth}
		\includegraphics[width=0.7\linewidth]{images/h2_temporal_news2}
		\href{https://www.rechargenews.com/energy-transition/proposed-stringent-eu-rules-on-green-hydrogen-would-put-the-brakes-on-development-/2-1-1223746}{Hydrogeninsight, May 2022}
		\includegraphics[width=0.7\linewidth]{images/h2_temporal_news3}
		\href{https://www.hydrogeninsight.com/policy/far-from-perfect-strict-rules-in-new-delegated-act-will-make-green-h2-projects-more-expensive-hydrogen-europe/2-1-1403241}{Hydrogeninsight, February 2023}
		
	\end{column}
	\end{columns}
\end{frame}
\begin{frame}{Temporal matching: definition}

  -- maybe we show some teasing screenshot before showing this slide with math, e.g., industry figting for easier regulation.

  \alert{Hourly matching} is modelled with a constraint (\ref{eqn:CFE}), 
  which matches electricity demand for eletrolysis process with generation of renewable resources on an hourly basis. 

  \vspace{0.1cm}
  \begin{equation}
  \sum_{r\in R} g_{r,t} + \sum_{s\in S} \left(\bar{g}_{s,t} - \ubar{g}_{s,t}\right) - ex_t = d_t
  \label{eqn:CFE}
  \end{equation}
  \vspace{0.1cm}
  
  \noindent\fbox{%
    \parbox{\textwidth}{%
  Here eq. 2 from https://iopscience.iop.org/article/10.1088/1748-9326/ad2239/pdf \\
  -- maybe here we say along the lines that the main fight in the Delegated Act was for this equation. \\
  -- explain the equation\\ 
  -- Now, what are the hidden issues? let's talk about the first point -- flexibility. \\
  }}
  \source{Zeyen et al. (2023): Temporal regulation \\ of renewable supply for electrolytic hydrogen}

  Elisabeth?

\end{frame}


\begin{frame}{Temporal matching: flexibility changes everything}
	\centering
	\includegraphics[width=0.6\linewidth, clip, trim={0cm 5cm 0cm 5cm}]{images/store_flexibility_v2}
	\begin{columns}[t]
		\begin{column}{0.5\textwidth}
			  	\centering
			\textbf{Emissions} \\
  	\includegraphics[width=1\linewidth, clip, trim={5cm 4cm 5cm 4cm}]{images/emissions_flexibility}
  	\end{column}
\begin{column}{0.5\textwidth}
		\centering
	\textbf{Hydrogen production costs} \\
	\includegraphics[width=1\linewidth, clip, trim={5cm 4cm 5cm 4cm}]{images/costs_flexibility}
\end{column}
\end{columns}
	\centering
    \alert{Flexible} operation \alert{reduces} emissions and costs, but \alert{constant} operations \alert{increase} them.
    
  \source{Zeyen et al. (2023): Temporal regulation \\ of renewable supply for electrolytic hydrogen} 

\end{frame}


\begin{frame}{Temporal matching: higher excess sales reduce emissions}

  	\begin{columns}[t]
  	\begin{column}{0.5\textwidth}
  		\centering
  		\textbf{Annual} \\
  		\includegraphics[width=1\linewidth]{images/annual_excess}
  	\end{column}
  	\begin{column}{0.5\textwidth}
  		\centering
  		\textbf{Hourly} \\
  		\includegraphics[width=1\linewidth]{images/hourly_excess}
  	\end{column}
  \end{columns}
  \source{Zeyen et al. (2023): Temporal regulation \\ of renewable supply for electrolytic hydrogen}

\end{frame}


\begin{frame}{Temporal matching: (CFE story)}

  Here, the \alert{hourly matching} constraint is extended to include electricity consumption from the regional grid:

  \vspace{0.1cm}
  \begin{equation}
  \sum_{r\in CFE, t\in T} g_{r,t} + \sum_{s\in STO, t\in T} \left(\bar{g}_{s,t} - \ubar{g}_{s,t}\right) - \sum_{t\in T} ex_t + \textcolor{red}{\sum_{t\in T} CFE_t \cdot im_t} \geq x \cdot \sum_{t\in T} d_t
  \label{eqn:CFE}
  \end{equation}
  \vspace{0.1cm}
  \source{Riepin \& Brown (2023): https://zenodo.org/records/7180098}

  \noindent\fbox{%
    \parbox{\textwidth}{%
    The \alert{grid CFE factor} $CFE_t$ in eq. (\ref{eqn:CFE}) defines the share of 
    carbon-free electricity in grid imports. \\
    -- depends how you scope out CFE calculation \\
    -- storage tracing \\
    -- CFE isn't a parameter but a variable \\
  }}
  Iegor

\end{frame}


\section{II Pillar - Geographical matching}
\begin{frame}{II Pillar - Geographical matching}
  Iegor?
  \includegraphics[width=13cm]{images/go_leak.png}
  \source{https://www.ffe.de/en/publications/norway-and-the-double-marketing-of-renewable-energies/}
\end{frame}



\begin{frame}{Geographical matching}

  \includegraphics[width=8cm]{images/microsoft.png}
  \source{datacenterdynamics "Microsoft signs 24/7 nuclear power deal with Constellation for Boydton data center" (June 30, 2023)}

\end{frame}

\begin{frame}{Geographical matching}

  \includegraphics[width=13cm]{images/microsoft2.png}

\end{frame}
  

\begin{frame}{Geographical matching}
\begin{columns}
	\begin{column}{0.4\textwidth}
		\includegraphics[width=5cm, clip, trim={0 0.1cm 0 0}]{images/biding_zones_eu.png}
		\source{\href{https://acer.europa.eu/sites/default/files/documents/en/Documents/Presentations\%20Webinars/20210624_Public_webinar_alternative_BZ_configurations.pdf}{ACER, 2021}}
	\end{column}
	\begin{column}{0.6\textwidth}
		Elisabeth?
		explain how geographical matching is defined IRL\\
		explain that it's relatively easy to implement in models\\
		geographical matching is than both string and unstict, depending how you look at it\\
		in some zones, plenty of congestions!\\
	\end{column}
\end{columns}
\end{frame}

\section{III Pillar - Additionality}
\begin{frame}{III Pillar - Additionality}

  Iegor starts? 

  -- start with the fact that it's complex and messy both in real world and in models\\
  -- define it (over MW and MWh)\\
  -- as a funny reference, we can re-use the Microsoft PPA deal example with existing nuclear power.

\end{frame}


\begin{frame}{Additionality: definition challenges}

  Iegor or Elisabeth?

  -- additionality in the real world: RED definition, Google definition, DA definition\\

\end{frame}


\begin{frame}{Additionality: how to model a background grid 1/2?}

  Iegor

  \begin{columns}
    \begin{column}{7cm}
      \includegraphics[width=7.5cm]{images/10-2025-DE-p3-ci_emisrate.pdf}
    \end{column}
    \begin{column}{7cm}
      \includegraphics[width=7.5cm]{images/ci_emisrate.pdf}
    \end{column}
  \end{columns}

  Scenario: Germany -- 2025 -- 24/7 CFE with 10\% participation of C\&I sector -- baseload profile -- all costs from \href{https://ens.dk/en/our-services/technology-catalogues}{DEA 2025}. Left: Constrained RES expansion (the Easter Package), Right: unconstrained RES expansion.
\end{frame}


\begin{frame}{Additionality: how we model a background grid 2/2?}

  Elisabeth

  -- here: cooptimisation vs sequential optimisation, results and thoughts about all it.

  -- conclude with reference to Lissy's paper on additionality

\end{frame}

\section{Take Aways}
\begin{frame}{Take aways (first version)}

\begin{columns}
\begin{column}{7cm}
  \begin{enumerate}
    \item Do not take results for granted, try to challenge everything! \\
    \enquote{We must fight our biases to see the solutions} (Auke Hoekstra)
    \item Talk to others in research community, share your concerns! \\
    \enquote{A problem shared is a problem halved} (Old saying)
  \end{enumerate}
\end{column}
\begin{column}{8cm}
  \includegraphics[width=8cm]{images/alice.png}
  \source{ChatGPT4 hallucination, 2024}
\end{column}
\end{columns}


\end{frame}

%----------------------------------------
%----------------------------------------

\begin{frame}\frametitle{\quad}

  {\Large
  \alert{Contacts, Resources, Acknowledgements}
  }

  \vspace{0.1cm}
    
  {\bf References:}
  \hrefc{https://iopscience.iop.org/article/10.1088/1748-9326/ad2239}{Temporal regulation of renewable supply for electrolytic hydrogen (2023)}\\
  {\bf References:} \hrefc{https://irieo.github.io/247cfe.github.io/}{More about the 24/7 CFE research project (2022-2024)}

  {\bf Code:} This work done in a spirit of open and reproducible research: \\
  \faUnlock~code:
  \hrefc{https://github.com/PyPSA/247-cfe}{github.com/PyPSA/247-cfe} \\
  \faUnlock~code: \hrefc{https://zenodo.org/records/8324521}{https://zenodo.org/records/8324521}
  
  \vspace{.1cm}
  {\bf Copyright:} Unless otherwise stated, graphics and text are Copyright \copyright E.Z. and I.R. 2024. \\
  This work is licensed under a \href{https://creativecommons.org/licenses/by/4.0/}{CC BY 4.0}.  {\footnotesize \ccby} 

  \vspace{.1cm}
  {\bf Send an email:} \\
  Dr. Elisabeth Zeyen, e.zeyen@tu-berlin.de\\
  Dr. Iegor Riepin, iegor.riepin@tu-berlin.de \\
  
\end{frame}


%%%%%%%%%%%%%%%%%%%%%%%%%%%%%%%%%%%%%%%%%%%%%%%%%%%%%%
\end{document}
%%%%%%%%%%%%%%%%%%%%%%%%%%%%%%%%%%%%%%%%%%%%%%%%%%%%%%